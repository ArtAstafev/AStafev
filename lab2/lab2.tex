\documentclass{beamer}
\usetheme{Boadilla}
\title{Начало криптографии}
\author{Артем Астафьев}
\institute{ИФМНиИТ|БФУ им.И.Канта}
\date{\today}
\usepackage[utf8]{inputenc}
\usepackage[russian]{babel}

\begin{document}
	\begin{frame}
		\titlepage
	\end{frame}


\begin{frame}
	\frametitle{Определения}
	\begin{itemize}
		\item Криптография — наука о методах обеспечения конфиденциальности, целостности данных, аутентификации , шифрования.
		 
		\item Криптосистема — это завершённая комплексная модель, способная производить двусторонние криптопреобразования над данными произвольного объёма и подтверждать время отправки сообщения, обладающая механизмом преобразования паролей, ключей и системой транспортного кодирования. 
		
	\end{itemize}
	
\end{frame}

\begin{frame}
	\frametitle{Построение криптосистемы}
	Для построения криптосистемы нужно проити определенные шаги:
	
	\begin{itemize}
		\item[1] Исходный текст(длинная цепочка символов алфавита) "разрезается" на конечные последовательности символов постоянной длины, называемые единичные сообщения.
		
		\item[2] единичные сообщения маркируются при помощи объектов, удобных для вычислений(Например при помощи натуральных чисел)
		
		\item[3] P - множество единичных сообщений, которые возможны.
		E - множество зашифрованных единичных сообщений, которые возможны.К замаркированным единичным сообщениям применяется шифрующее преобразование, которое всякому единичному сообщению x $\in$ P сопостовляет однозначно определенное единичное сообщение y $\in$ E.
		
		Для дешифровки применяется дешифрующие преобразование, сопостовляющее единичному сообщению y$\in$E однозначно определенное сообщение x$\in$P
		
	\end{itemize}
	
\end{frame}

\end{document}

