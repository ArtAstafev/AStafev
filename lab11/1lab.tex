\documentclass[11pt]{article}
\usepackage{amsmath,amssymb,amsthm}
\usepackage{algorithm}
\usepackage[noend]{algpseudocode} 

%---enable russian----

\usepackage[utf8]{inputenc}
\usepackage[russian]{babel}

% PROBABILITY SYMBOLS
\newcommand*\PROB\Pr 
\DeclareMathOperator*{\EXPECT}{\mathbb{E}}


% Sets, Rngs, ets 
\newcommand{\N}{{{\mathbb N}}}
\newcommand{\Z}{{{\mathbb Z}}}
\newcommand{\R}{{{\mathbb R}}}
\newcommand{\Zp}{\ints_p} % Integers modulo p
\newcommand{\Zq}{\ints_q} % Integers modulo q
\newcommand{\Zn}{\ints_N} % Integers modulo N

% Landau 
\newcommand{\bigO}{\mathcal{O}}
\newcommand*{\OLandau}{\bigO}
\newcommand*{\WLandau}{\Omega}
\newcommand*{\xOLandau}{\widetilde{\OLandau}}
\newcommand*{\xWLandau}{\widetilde{\WLandau}}
\newcommand*{\TLandau}{\Theta}
\newcommand*{\xTLandau}{\widetilde{\TLandau}}
\newcommand{\smallo}{o} %technically, an omicron
\newcommand{\softO}{\widetilde{\bigO}}
\newcommand{\wLandau}{\omega}
\newcommand{\negl}{\mathrm{negl}} 

% Misc
\newcommand{\eps}{\varepsilon}
\newcommand{\inprod}[1]{\left\langle #1 \right\rangle}

 
\newcommand{\handout}[5]{
  \noindent
  \begin{center}
  \framebox{
    \vbox{
      \hbox to 5.78in { {\bf Научно-исследовательская практика} \hfill #2 }
      \vspace{4mm}
      \hbox to 5.78in { {\Large \hfill #5  \hfill} }
      \vspace{2mm}
      \hbox to 5.78in { {\em #3 \hfill #4} }
    }
  }
  \end{center}
  \vspace*{4mm}
}

\newcommand{\lecture}[4]{\handout{#1}{#2}{#3}{Scribe: #4}{Элементарная теория чисел #1}}

\newtheorem{theorem}{Теорема}
\newtheorem{lemma}{Лемма}
\newtheorem{definition}{Определение}
\newtheorem{corollary}{Следствие}
\newtheorem{fact}{Факт}

% 1-inch margins
\topmargin 0pt
\advance \topmargin by -\headheight
\advance \topmargin by -\headsep
\textheight 8.9in
\oddsidemargin 0pt
\evensidemargin \oddsidemargin
\marginparwidth 0.5in
\textwidth 6.5in

\parindent 0in
\parskip 1.5ex

\begin{document}

\lecture{}{Лето 2020}{}{Астафьев Артем}


\section{Наибольший Общий Делитель}

\subsection{Анализ}


Для иллюстрации последнего следствия заметим, 
$\text{НОД}(-12, 30)=6$ и 

\[\text{НОД}(-12/6, 30/6)=\text{НОД}(-2, 5)=1,\]

как оно и должно быть.
    Это неверно, не добавляя дополнительного условия, что
$a|c$ и $b|c$ вместе дают $ab|c$. 
\setcounter{corollary}{1}
\begin{corollary}
Если $a|c$ и $b|c$, и $\text{НОД}(a,b)=1$, знаит $ab|c$.
\begin{proof}
	Поскольку $a|c$ и $b|c$, $r$ и $s$ целые числа то  можно найти такое $c = ar=bs$.Теперь взаимосвязь $\text{НОД}(a, b)=1$ позволяет нам написать $1=ax+by$ для некоторого набора  целых чисел $x$ и $y$. Умножим последнее уравнение на $c$, похоже,что
	
		\[c = c \cdot 1 = c(ax+by)=acx+bcy\]
		
если соответствующие замены теперь сделаны с правой стороны, то

	\[c=a(bs)x+b(ar)y=ab(sx+ry)\]
	
или, как заявление о невозможности, $ab|c$.
\end{proof}
\end{corollary}
Наш следующее следствие кажется довольно небольшим, но имеет фундаментальное значение

\setcounter{theorem}{4}
\begin{theorem}[Лемма Евклида]
	Если $a|bc$, и $\text{НОД}(a,b)=1$,значит $a|c$.
\end{theorem}
\begin{proof}
	Мы снова начнем с Теоремы 3(из этой главы), напишем $1 = ax+by$, где $x$ и $y$ целые числа. Умножение этого уравнение на $c$ дает

		\[c = 1 \cdot c=(ax+by)c=acx+bcy\]

    Потому что $a|ac$ и $a|bc$, из этого следует что $a|(acx+bcy)$,что можно представить как $a|c$.
\end{proof}
Если $a$ и $b$ не являются относительно простыми, то вывод леммы Евклида может оказаться несостоятельным.

Последующая теорема часто служит определением $\text{НОД}(a,b)$. Преимущество использования его в качестве определения состоит в том, что здесь не задействованы отношения порядка. Таким образом, он может быть использован в алгебраических системах, не имеющих отношения порядка. 

\begin{theorem}
	Пусть $a$,$b$ целые числа,не равные нулю.Для положительного целого числа $d$,$d=\text{НОД}(a,b)$ тогда и только тогда когда
\begin{enumerate}
    \item[(1)] $d|a$ и $d|b$
	\item[(2)] если $c|a$ и $c|b$,то $c|d$.
\end{enumerate}
\end{theorem}
\begin{proof}
	Для начала предположим, что $d=\text{НОД}(a,b)$.Конечно, $d|a$ и $d|b$, так что (1) имееи место.В свете Теоремы 3(из этой главы),$d$ выражается как $d=ax+by$ для некоотрых целых чисел $x$ и $y$.
	Таким образом, если $c|a$, значит $c|(ax+by)$,или вернее $c|d$.
	Короче говоря, утверждение (2) имеет место.Наоборот, пусть $d$ любое положительное число,удовлетворяющее указаым условиям.
	Учитывая любой общий делитель $c$ из $a$ и $b$, мы имеем $c|d$ из гепотезы (2).Импликация заключается в том $d \ge c$, и вследствии $d$ это $\text{НОД}(a,b)$.
\end{proof}

\subsection{вопросы}
\begin{enumerate}
\item Если $a|b$, покожите,что $(-a)|b$, $a|(-b)$, $(-a)|(-b)$.
\item Даны целые числа $a$,$b$,$c$, убедитесь, что
\begin{enumerate}
    \item[(а)] Если $a|b$, то $a|bc$;
	\item[(b)] Если  $a|b$ и,значит $a|c$ $a^{2}|bc$;
	\item[(c)] $a|b$ тогда и только тогда,когда $ac|bc$, где $c \neq 0$.
\end{enumerate}

\item доказать или опровергнуть: если $a|(b+c)$, тогда или $a|b$ или $a|c$.
\item Докажите,что для любого целого числа $a$, одно из чисел $a + 2$, $a + 4$ делится на $3$.
\item Для произвольного целого числа $a$, установить, что $2|a(a+1)$ пока $3|a(a+1)(a+2)$.
\item Для $n\ge 1$, использую индукцию докажите,что
\begin{enumerate}
	\item[(а)] $7$ делится на $2^{3n}-1$ и $8$ делится $3^{2n}+7$
	\item[(b)] $2^{n}+(-1)^{n+1}$ делится на $3$.
\end{enumerate}
\item Докажите что
\begin{enumerate}
	\item[(а)] Сумма квадратов двух нечетных чисел не может быть полным квадратом;
	\item[(b)] Произведение четырёх последовательных целых чисел на один меньше, чем полный квадрат.
\end{enumerate}
\item Установите, что разница двух последователных кубов никогда не делится на $2$.
\item Для ненулевого целого числа а покажите, что $\text{НОД}(a,0)=|a|$, $\text{НОД}(a,a)=|a|$, и $\text{НОД}(a,1)=1$.
\item  Если $a$ и $b$ целые числа,не равные нулю, убедитесь, что

	\[\text{НОД}(a,b)=\text{НОД}(-a,b)=\text{НОД}(a,-b)=\text{НОД}(-a,-b).\]


\item Докажите, что для положительного целого число $n$ и любого целого числа $a$, $\text{НОД}(a,a+n)$ делится на $n$;следовательно, $\text{НОД}(a,a+1)=1$.

\item Даны целые числа $a$ и $b$,докажите,что
\begin{enumerate}
	\item[(а)] Существуют целые числа $x$ и $y$,для которых $c=ax+by$ тогда и только тогда,когда $\text{НОД}(a,b)|c$;
	\item[(b)] Если существуют целые числа $x$ и $y$, для которых $ax+by=\text{НОД}(a,b)$,значит $\text{НОД}(x,y)=1$.
\end{enumerate}
\item Докажите: Произведение любых трех последовательных целых чисел делится на $6$;произвеление любых четырех последовательных целых чисел делится на $24$;произведение любых $5$ последовательных целых чисел делится на $120$.
\item Установите каждое из приведенных ниже утверждений:
\begin{enumerate}
	\item[(а)] Если а неетное целое число, значит $24|a(a^{2}-1$).
	\item[(b)] Если $a$ и $b$ нечетные целые числа, значит $8|a(a^{2}-b^{2}$).
	\item[(c)] Если $а$ целое число не делющиеся на $2$ или $3$, значит $24|a(a^{2}+23$).
	\item[(d)] Если $а$ произвольное целое число, значит $360|a^{2}(a^{2}-1)(a^{2}-4)$.
\end{enumerate}

\item Убедитесь,что следующие свойства наибольщего общего делителя имеют место:
\begin{enumerate}
	\item[(a)] Если $\text{НОД}(a,b)=1$ и $\text{НОД}(a,c)=1$, значит $\text{НОД}(a,bc)=1$.
	\item[(b)] Если $\text{НОД}(a,b)=1$ и $c|a$, значит $\text{НОД}(b,c)=1$.
	\item[(c)] Если $\text{НОД}(a,b)=1$ ,значит $\text{НОД}(ac,b)=\text{НОД}(c,b)$.
	\item[(d)] Если $\text{НОД}(a,b)=1$ и $c|a+b$, значит $\text{НОД}(a,c)=\text{НОД}(b,c)=1$.
\end{enumerate}

\end{enumerate}

\section{Алгоритм Евклида}
\subsection{Анализ}
Наибольший общий делитель двух целых чисел можно найти, перечислив все их возможные делители и выбрав наибольший общий для каждого из них; но это слишком много для больших чисел. Более эффективный процесс, включающий повторное применение алгоритма деления, приведен в седьмой книге
Элементы.
Хотя есть исторические свидетельства того, что этот метод предшествовал Евклиду, сегодня он называется
алгоритм Евклида

 Евклидов алгоритм может быть описан следующим образом пусть $a$ и $b$-два целых числа,наибольший общий делитель которых нужно найти.Так как $\text{НОД}(|a|,|b|)=\text{НОД}(a,b)$,нет ничего плохого в предположении,что $a\ge b > 0$.
 Первый шаг-применить алгоритм деления к $a$ и $b$, чтобы получить
 
 	\[a=q_1b+r_1, 0\le r_1<b\]
 
 Если так случится,что $r_1=0$,значит $b|a$ и $\text{НОД}(a,b)=b$. Когда $r_1 \neq 0$,
 делим $b$ на $r_1$ для получения целых чисел $q_2$ и $r_2$ удовлетворяющий
 
 	\[b = q_2r_1 + r_2, 0\le r_2<r_1\]
 
 Если $r_2=0$, значит мы останавливаемся;в противном случае действуем так же, как и раньше, чтобы получить
 
 	\[r_1 = q_3r_2 + r_3, 0\le r_3<r_2\]
 
этот процесс деления продолжается до тех пор, пока не появится некоторый нулевой остаток, скажем, на $(n+1)$ шаге,где $r_n-1$ делит $r_n$(нулевой остаток возникает рано или поздно, потому что убывающая последовательность $b > r_1 > r_2 > ... \ge 0$ не может содержать более $b$ целых чисел).

В результате получается следующая система уравнений:
\begin{center}
	$a=q_1 b+r_1$,	$0\le r_1<b$
	
	$b = q_2r_1 + r_2$, $0\le r_2<r_1$
	
	$r_1$ = $q_3$$r_2$ + $r_3$, $0\le r_3$<$r_2$

    $\vdots$
    
    $r_n-2 = q_nr_n + r_n$, $0\le r_n<r_n-1$
    
    $r_n-1 = q_n+1r_n + 0$
\end{center}
Мы утверждаем, что $r_n$ последний ненулевой остаток, который появляется таким образом,равным $\text{НОД}(a,b)$.Наше доказательство основано на приведенной ниже лемме.
\begin{lemma}
	Если $a=qb+r$,значит $\text{НОД}(a,b)=\text{НОД}(b,r)$.
\end{lemma}
\nocite{Burton}
\bibliographystyle{plain}
\bibliography{b.bib}

\end{document}
