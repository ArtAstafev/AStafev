\documentclass[11pt]{article}
\usepackage{amsmath,amssymb,amsthm}
\usepackage{algorithm}
\usepackage[noend]{algpseudocode} 

%---enable russian----

\usepackage[utf8]{inputenc}
\usepackage[russian]{babel}

% PROBABILITY SYMBOLS
\newcommand*\PROB\Pr 
\DeclareMathOperator*{\EXPECT}{\mathbb{E}}


% Sets, Rngs, ets 
\newcommand{\N}{{{\mathbb N}}}
\newcommand{\Z}{{{\mathbb Z}}}
\newcommand{\R}{{{\mathbb R}}}
\newcommand{\Zp}{\ints_p} % Integers modulo p
\newcommand{\Zq}{\ints_q} % Integers modulo q
\newcommand{\Zn}{\ints_N} % Integers modulo N

% Landau 
\newcommand{\bigO}{\mathcal{O}}
\newcommand*{\OLandau}{\bigO}
\newcommand*{\WLandau}{\Omega}
\newcommand*{\xOLandau}{\widetilde{\OLandau}}
\newcommand*{\xWLandau}{\widetilde{\WLandau}}
\newcommand*{\TLandau}{\Theta}
\newcommand*{\xTLandau}{\widetilde{\TLandau}}
\newcommand{\smallo}{o} %technically, an omicron
\newcommand{\softO}{\widetilde{\bigO}}
\newcommand{\wLandau}{\omega}
\newcommand{\negl}{\mathrm{negl}} 

% Misc
\newcommand{\eps}{\varepsilon}
\newcommand{\inprod}[1]{\left\langle #1 \right\rangle}

 
\newcommand{\handout}[5]{
  \noindent
  \begin{center}
  \framebox{
    \vbox{
      \hbox to 5.78in { {\bf Научно-исследовательская практика} \hfill #2 }
      \vspace{4mm}
      \hbox to 5.78in { {\Large \hfill #5  \hfill} }
      \vspace{2mm}
      \hbox to 5.78in { {\em #3 \hfill #4} }
    }
  }
  \end{center}
  \vspace*{4mm}
}

\newcommand{\lecture}[4]{\handout{#1}{#2}{#3}{Scribe: #4}{Элементарная теория чисел #1}}

\newtheorem{theorem}{Теорема}
\newtheorem{lemma}{Лемма}
\newtheorem{definition}{Определение}
\newtheorem{corollary}{Следствие}
\newtheorem{fact}{Факт}

% 1-inch margins
\topmargin 0pt
\advance \topmargin by -\headheight
\advance \topmargin by -\headsep
\textheight 8.9in
\oddsidemargin 0pt
\evensidemargin \oddsidemargin
\marginparwidth 0.5in
\textwidth 6.5in

\parindent 0in
\parskip 1.5ex

\begin{document}

\lecture{}{Лето 2020}{}{Астафьев Артем}


\section{Наибольший Общий Делитель}

\subsection{Анализ}


Для иллюстрации последнего следствия заметим, 
НОД(-12, 30)=6 и 
\begin{center}
НОД(-12/6, 30/6)=НОД(-2, 5)=1,
\end{center}
как оно и должно быть.
    Это неверно, не добавляя дополнительного условия, что
a|c и b|c вместе дают ab|c. 
\setcounter{corollary}{1}
\begin{corollary}
Если a|c и b|c, и НОД(a,b)=1, знаит ab|c.
\begin{proof}
	Поскольку a|c и b|c, r и s целые числа то  можно найти такое с = ar=bs.Теперь взаимосвязь НОД(a, b)=1 позволяет нам написать 1=ax+by для некоторого набора  целых чисел x и y Умножим последнее уравнение на с,похоже,что
	\begin{center}
		с = с*1 = с(ax+by)=acx+bcy
	\end{center}
если соответствующие замены теперь сделаны с правой стороны, то
\begin{center}
	c=a(bs)x+b(ar)y=ab(sx+ry)
\end{center}
или, как заявление о невозможности, ab|c.
\end{proof}
\end{corollary}
Наш следующее следствие кажется довольно небольшим, но имеет фундаментальное значение

\setcounter{theorem}{4}
\begin{theorem}
	(Лемма Евклида).Если a|bc, и НОД(a,b)=1,значит a|c.
\end{theorem}
\begin{proof}
	Мы снова начнем с Теоремы 3(из этой главы), напишем 1 = ax+by, где х и у целые числа. Умножение этого уравнение на с дает
	\begin{center}
		c = 1*c=(ax+by)c=acx+bcy
	\end{center}
    Потому что a|ac и a|bc, из этого следует что a|(acx+bcy),что можно представить как a|c.
\end{proof}
Если a и b не являются относительно простыми, то вывод леммы Евклида может оказаться несостоятельным.

Последующая теорема часто служит определением НОД (a,b). Преимущество использования его в качестве определения состоит в том, что здесь не задействованы отношения порядка. Таким образом, он может быть использован в алгебраических системах, не имеющих отношения порядка. 

\begin{theorem}
	Пусть a,b целые числа,не равные нулю.Для положительного целого числа d,d=НОД(a,b) тогда и только тогда когда
	\begin{flushleft}
		(1) d|a и d|b
	\end{flushleft}
	\begin{flushleft}
	(2) если c|a и c|b,то c|d.
	\end{flushleft}
\end{theorem}
\begin{proof}
	Для начала предположим, что d=НОД(a,b).Конечно, d|a и d|b, так что (1) имееи место.В свете Теоремы 3(из этой главы),d выражается как d=ax+by для некоотрых целых чисел x и y.
	Таким образом, если c|a, значит c|(ax+by),или вернее c|d.
	Короче говоря, утверждение(2) имеет место.Наоборот, пусть d любое положительное число,удовлетворяющее указаым условиям.
	Учитывая любой общий делитель c из а и b, мы имеем c|d из геротезы (2).Импликация заключается в том $d \ge c$, и  вследствии d это НОД(a,b).
\end{proof}

\subsection{вопросы}
\begin{enumerate}
\item Если a|b, покожите,что (-a)|b, a|(-b),(-a)|(-b)
\item Даны целые числа a,b,c, убедитесь, что
\begin{flushleft}
	(а)  Если a|b, то a|bc;
\end{flushleft}
\begin{flushleft}
	(b) Если  a|b и,значит a|c $a^{2}|bc$;
\end{flushleft}
\begin{flushleft}
	(c)  a|b тогда и только тогда,когда ac|bc, где $c \neq 0$.
\end{flushleft}

\item доказать или опровергнуть: если a|(b+c), тогда или a|b или a|c.
\item Докажите,что для любого целого числа а, одно из чисел а + 2, а + 4 делится на 3.
\item Для произвольного целого числа а, установить, что 2|a(a+1) пока 3|a(a+1)(a+2).
\item Для $n\ge 1$, использую индукцию докажите,что
\begin{flushleft}
	(а)  7 делится на $2^{3n}-1$ и 8 делится $3^{2n}+7$
\end{flushleft}
\begin{flushleft}
	(b) $2^{n}+(-1)^{n+1}$ делится на 3.
\end{flushleft}
\item Докажите что
\begin{flushleft}
	(а)  Сумма квадратов двух нечетных чисел не может быть полным квадратом;
\end{flushleft}
\begin{flushleft}
	(b)Произведение четырёх последовательных целых чисел на один меньше, чем полный квадрат.
\end{flushleft}
\item Установите, что разница двух последователных кубов никогда не делится на 2.
\item Для ненулевого целого числа а покажите, что НОД(а,0)=|a|,НОД(а,а)=|a|, и НОД(а,1)=1.
\item  Если a и b целые числа,не равные нулю, убедитесь, что
\begin{center}
	НОД(a,b)=НОД(-a,b)=НОД(a,-b)=НОД(-a,-b).
\end{center}

\item Докажите, что для положительного целого число n и любого целого числа a,НОД(a,a+n) делится на n;следовательно,НОД(а,а+1)=1.

\item Даны целые числа a и b,докажите,что
\begin{flushleft}
	(а) Существуют целые числа x и y,для которых с=ах+by тогда и только тогда,когда НОД(а,b)|c;
\end{flushleft}
\begin{flushleft}
	(b) Если существуют целые числа х и у, для которых ах+by=НОД(a,b),значит НОД(х,у)=1.
\end{flushleft}
\item Докажите: Произведение любых трех последовательных целых чисел делится на 6;произвеление любых четырех последовательных целых чисел делится на 24;произведение любых 5 последовательных целых чисел делится на 120.
\item Установите каждое из приведенных ниже утверждений:
\begin{flushleft}
	(а) Если а неетное целое число, значит 24|a($a^{2}-1$).
\end{flushleft}
\begin{flushleft}
	(b) Если а и b нечетные целые числа, значит 8|a($a^{2}-b^{2}$).
\end{flushleft}
\begin{flushleft}
	(c) Если а целое число не делющиеся на 2 или 3, значит 24|a($a^{2}+23$).
\end{flushleft}
\begin{flushleft}
	(d) Если а произвольное целое число, значит 360|$a^{2}(a^{2}-1)(a^{2}-4)$.
\end{flushleft}

\item Убедитесь,что следующие свойства наибольщего общего делителя имеют место:
\begin{flushleft}
	(a) Если НОД(a,b)=1 и НОД(a,c)=1, значит НОД(a,bc)=1.
\end{flushleft}
\begin{flushleft}
	(b) Если НОД(a,b)=1 и с|a, значит НОД(b,c)=1.
\end{flushleft}\begin{flushleft}
	(c) Если НОД(a,b)=1 ,значит НОД(ac,b)=НОД(c,b).
\end{flushleft}
\begin{flushleft}
	(d) Если НОД(a,b)=1 и c|a+b, значит НОД(a,c)=НОД(b,c)=1.
\end{flushleft}

\end{enumerate}

\section{Алгоритм Евклида}
\subsection{Анализ}
Наибольший общий делитель двух целых чисел можно найти, перечислив все их возможные делители и выбрав наибольший общий для каждого из них; но это слишком много для больших чисел. Более эффективный процесс, включающий повторное применение алгоритма деления, приведен в седьмой книге
Элементы.
Хотя есть исторические свидетельства того, что этот метод предшествовал Евклиду, сегодня он называется
алгоритм Евклида

 Евклидов алгоритм может быть описан следующим образом пусть a и b-два целыхчисла,наибольший общий делитель которых желателен.Так как НОД(|a|,|b|)=НОД(a,b),нет ничего плохого в предположении,что $a\ge b$ > 0.
 Первый шаг-применить алгоритм деления к a и b, чтобы получить
 \begin{center}
 	$a=q_1$b+$r_1$,	$0\le r_1$<b
 \end{center}
 Если так случится,что $r_1$=0,значит b|a и НОД(a,b)=b. Когда $r_1 \neq 0$,
 делим b на $r_1$ для получения целых чисел $q_2$ и $r_2$ удовлетворяющий
 \begin{center}
 	b = $q_2$$r_1$ + $r_2$, $0\le r_2$<$r_1$
 \end{center}
 Если $r_2$=0, значит мы останавливаемся;в противном случае действуйте так же, как и раньше, чтобы получить
 \begin{center}
 	$r_1$ = $q_3$$r_2$ + $r_3$, $0\le r_3$<$r_2$
 \end{center}
этот процесс деления продолжается до тех пор, пока не появится некоторый нулевой остаток, скажем, на (n+1) шаге,где $r_n-1$ делит $r_n$(нулевой остаток возникает рано или поздно, потому что убывающая последовательность $b > r_1 > r_2 > ... \ge 0$ не может содержать более b  целых чисел)

В результате получается следующая система уравнений:
\begin{center}
	$a=q_1$b+$r_1$,	$0\le r_1$<b
	
	b = $q_2$$r_1$ + $r_2$, $0\le r_2$<$r_1$
	
	$r_1$ = $q_3$$r_2$ + $r_3$, $0\le r_3$<$r_2$

    ***
    
    $r_n-2$ = $q_n$$r_n$ + $r_n$, $0\le r_n$<$r_n-1$
    
    $r_n-1$ = $q_n+1$$r_n$ + 0
\end{center}
Мы утверждаем, что $r_n$ последний ненулевой остаток, который появляется таким образом,равным НОД(a,b).Наше доказательство основано на приведенной ниже лемме.
\begin{lemma}
	Если a=qb+r,значит НОД(a,b)=НОД(b,r).
\end{lemma}
\cite{Burton}
\bibliographystyle{plain}
\bibliography{b.bib}

\end{document}